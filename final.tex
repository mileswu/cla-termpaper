\documentclass[11pt]{article}
\setlength{\topmargin}{0in}
\setlength{\headheight}{0in}
\setlength{\headsep}{0in}
\setlength{\textheight}{8.7in}
\setlength{\textwidth}{6.5in}
\setlength{\oddsidemargin}{0in}
\setlength{\evensidemargin}{0in}
\setlength{\parindent}{0.3in}
%\setlength{\parskip}{0.10in}

\begin{document}

\title{Rape in Greek Myth}
\author{Miles Wu (Preceptor: Dawn LaValle)}
\maketitle

\begin{abstract}

\end{abstract}

\newpage

\section{Introduction}
The motif of rape occurs frequently across Classical Greek myths, but it does not always appear under its modern guise.
To us, while the word `rape' refers to the lack of consent on the woman's part to engaging in sexual intercourse, in many Greek myths it does not seem to mean that.
For example, in the infamous myth of the Rape of Helen of Troy, there is little indication that Helen is being forced against her will to follow and make love to Paris, as Aphrodite has caused her to become infatuated with Paris, yet the myth is still often called a rape.

In an attempt to explain this discrepancy between the meaning of the word `rape' in modern translations of Greek myth and its current contemporary meaning, I look to the etymological root of the English word `rape', which is a Latin word, \emph{raptus} (the verb being \emph{rapio}).
This means `seized' or `taken by force' and is often used to refer to physical objects being stolen.
While it can have our modern meaning of sexual violation, this was very uncommon as another Latin word was used instead (\emph{stuprum}).
Perhaps in these Greek myths `rape' takes on \emph{raptus}' more common meaning, namely that of stealing.
This explanation implies that the women involved in these myths are treated as the property of a male and their consent is not considered at all in labeling their abduction a `rape'.

In this paper I will look at a couple of depictions of rape in Greek myth to see if this view of `rape' as stolen property is a more suitable and accurate perspective from which to look at and understand the stories, compared to the more standard view of rape as a sexual violation.
In particular I will be examining the primary sources of two famous mythological rapes, Helen of Troy and The Rape of Persephone.

\section{Helen of Troy}
See what passages in particular talk about the removal of Helen to Troy and
how they talk about it.

Does Helen have an opinion about what her ``rape" means?

Do the other characters like Priam or the other men of Troy in Book 3 of the
Iliad?

 In the Iliad, the pretty girls that were captured near Troy were given to the better fighters as "prizes", for example. This is obvious in Book 1.
 

\section{The Rape of Persephone}


\section{Conclusion}
Fsfsdfs \cite{iliad}.


\newpage
\begin{thebibliography}{9}
	\bibitem{iliad}
		Iliad.
	\bibitem{odyssey}
		Odyssey.
\end{thebibliography}

\end{document}
