\documentclass[11pt]{article}
\setlength{\topmargin}{0in}
\setlength{\headheight}{0in}
\setlength{\headsep}{0in}
\setlength{\textheight}{8.7in}
\setlength{\textwidth}{6.5in}
\setlength{\oddsidemargin}{0in}
\setlength{\evensidemargin}{0in}
\setlength{\parindent}{0.3in}
%\setlength{\parskip}{0.10in}

\begin{document}

\title{Rape in Greek Myth}
\author{Miles Wu (Preceptor: Dawn LaValle)}
\maketitle

\begin{abstract}

\end{abstract}

\newpage

\section{Introduction}
The motif of rape occurs frequently across Classical Greek myths, but it does not always appear under its modern guise.
To us, while the word `rape' refers to the lack of consent on the woman's part to engaging in sexual intercourse, in many Greek myths it does not seem to mean that.
For example, in the infamous myth of the Rape of Helen of Troy, many think that Helen was not  forced against her will to follow and make love to Paris, as Aphrodite has caused her to become infatuated with Paris, yet the myth is still often called a rape.

In an attempt to explain this discrepancy between the meaning of the word `rape' in modern translations of Greek myth and its current contemporary meaning, I look to the etymological root of the English word `rape', which is a Latin word, \emph{raptus} (the verb being \emph{rapio}).
This means `seized' or `taken by force' and is often used to refer to physical objects being stolen.
While it can have our modern meaning of sexual violation, this was very uncommon as another Latin word was used instead (\emph{stuprum}).
Perhaps in these Greek myths `rape' takes on \emph{raptus}' more common meaning, namely that of stealing.
This explanation implies that the women involved in these myths are treated as the property of a male and their consent is not considered at all in labeling their abduction a `rape'.

In this paper I will look at a couple of depictions of rape in Greek myth to see if this view of `rape' as stolen property is a more suitable and accurate perspective from which to look at and understand the stories, compared to the more standard view of rape as a sexual violation.
More generally I will also evaluate whether or not women in these myths are treated as objects of male ownership.
In particular I will be examining the primary sources of two famous mythological rapes, Helen of Troy and The Rape of Persephone.

\section{Helen of Troy}
%Establish consent
Helen's main appearance in the Iliad occurs in book three, where Paris offers to duel Menelaus in single combat to decide the fate of the whole war.
After Paris is saved from by Aphrodite, Aphrodite appears to Helen in the disguise of an old friend and tells her that ``Paris is calling for you'' \cite[book 3, line 450]{iliad}.
After Aphrodite's short description of how Paris is ``glistening in all his beauty'' and how he is waiting ``in the bedroom'' for her, it is said that these ``the heart in Helen's breast began to race'' \cite[book 3, line 456]{iliad}.
This makes it apparent that she is in love with Paris and freely wants to go visit him in his bedroom.
Once she reaches the bedroom, Paris ``led the way to bed'' and she ``went with him'' \cite[book 3, line 525]{iliad}.
There is no mention of Paris compelling her to go to his bed and the use of the word `led' implies that she followed him on her volition as it is not a word with a forceful nuance.
One could make the claim that use of the word `led' was a choice that the translator made and that the original Greek word did not have the same nuances of voluntariness.
In another translation of the Iliad, for that line it says ``his wife followed him'' instead \cite[book 3, line 448]{iliad-rieu}.
The use of the word `follow' here also has the same implication that she willingly went into bed with Paris.

%Prove previous consent
The above is only one instance where it is shown that she is not being forced by Paris, but we can infer from other passages that it applies generally.
For example, when Helen is talking to Priam, once again she uses the word `follow' by saying that she ``followed your son to Troy'' \cite[book 3, line 210]{iliad}.
After Paris is defeated by Menelaus, Helen initially does not want ``to share that coward's bed once more'' and says she'll ``never go back  again'', though she later relents as discussed in the preceding paragraph \cite[book 3, line 475]{iliad}.
The fact that she chooses not to go to make love to Paris, albeit only for a brief time, shows that she has free will in these matters and is not being forced.
Furthermore, since it says sharing his bed once more, it shows that they have had sex before and this is also supported by Paris saying that they ``loved in love on Rocky Island'' \cite[book 3, line 522]{iliad}.
%conclude consent
As a result I conclude that Helen was actually in love with Paris and willingly consented to sexual activities.

%blame game - what do others think
Another approach of determining whether Helen was consenting is to look at the issue of blame.
There are three obvious choices for people on whom the Trojan war can be blamed: Paris, Helen and Aphrodite.
Which choice makes the most sense depends on how one interprets the myth.
For instance, if Paris forcefully abducted Helen and raped her in our modern definition of rape then for him to shoulder the blame is obvious.
If Helen deserted her husband and willingly left for Troy, having fallen in love with Paris, then she should carry some of the blame.
Finally, according to many versions of the myth, Aphrodite caused Helen to fall in love with Paris as a reward for Paris choosing her in the divine beauty contest.
If one is to believe this, then perhaps Aphrodite she deserves some of the blame for meddling.
Looking at where characters in the Iliad and Odyssey place the blame gives clues as to which interpretation of the myth is most accurate.

When Odysseus visits the underworld in Book 11 of the Odyssey, he meets various slain Achaean heroes and they talk about the war.
Twice he refers to Helen in a negative light and sees her actions as the root cause of the war.
The first time he bemoans to Alcinous the loss of his comrades who ``perished all for the sake of a vile woman [Helen]'' \cite[book 11, line 384]{odyssey}.
The second time he expresses his pity at Agamemnon's misfortunes that have all come about ``through the schemes of women'' \cite[book 11, line 438]{odyssey}.
When he speaks of `women' in the previous quote, he is referring to two women in particular, Helen and Klytaimestra (Agamemnon's wife), because in the next sentence he says that ``many of us died for the sake of Helen, and ... Klytaimestra plotted treason against you'' \cite[book 11, line 438]{odyssey}.
It is clear from these that Odysseus places the blame of the Trojan war on Helen alone.

In book three of the Iliad, when Helen goes up onto the walls of the city to watch the duel between Menelaus and Paris, Priam greets Helen and makes it clear that he doesn't ``blame her'' \cite[book 3, line 198]{iliad}.
Although this is different from Odysseus, he still does not accuse Paris, but instead chooses to ``hold the gods to blame'' as ``they are the ones who brought this war'' \cite[book 3, line 200]{iliad}.

Were Paris to have abducted her against her will, everyone would surely blame him for the Trojan war as kidnapping and sexual violation are despicable.
However, in both the Iliad and Odyssey we do not find that everyone finds Paris responsible for the war and in fact most people find Helen to be at fault.
This implies strongly that Helen willingly left Menelaus for Paris, with or without Aphrodite's meddling.

%Transition to property
There seems to be no evidence in either the Iliad or the Odyssey for Helen being forced to go unwillingly to Troy with Paris, or for her being sexually violated by Paris.
Everything points to her actually being in love with Paris, following him voluntarily to Troy and consenting to sharing his bed.
This is obviously incompatible with the modern definition of `rape' and motivates our looking for evidence that another meaning for this word is being used.
As suggested in the Introduction, there is a linguistic justification for why `rape' might have the meaning of stolen property so we shall see if this is a more suitable meaning for the context of the myth.
If Helen is treated as stolen property 

%Property aspect
Hector when he is rebuking Paris for being a coward says that Menelaus is the ``man you robbed of his sumptuous, warm wife'' \cite[book 3, line 61]{iliad}.

%Helen respects duel (b3, l309): cos she is wondering aphrodite's trickery in luring

%HECTOR iliad, book 3 line 61: ROBBED
%Helen thinks very differently as she is sorry for war

%Book 1: women as prizes or trophies

%In the Odyssey, page 72, she wants to go back home. She has no choice in anything?

%In fact, Helen has only a few appearances in both the Iliad and Odyssey, which is surprising when one considers that she is the root cause of the entire war. WOMEN ARENT IMPORTANT IN GREEK MYTHS?


%Preceptor suggestions
%See what passages in particular talk about the removal of Helen to Troy and
%how they talk about it.

%Does Helen have an opinion about what her ``rape" means? Do the other characters like Priam or the other men of Troy in Book 3 of the Iliad?


\section{The Rape of Persephone}


\section{Conclusion}

%todo
% check one/we/I

\newpage
\bibliographystyle{alpha}
\begin{thebibliography}{9}
	\bibitem{iliad}
		Iliad (Lattimore).
	\bibitem{iliad-rieu}
		Iliad (Rieu).
	\bibitem{odyssey}
		Odyssey (Lattimore).
\end{thebibliography}

\end{document}